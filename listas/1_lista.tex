\documentclass[12pt]{article}
\usepackage[a4paper, left=2cm, right=2cm, top=2cm, bottom=2cm]{geometry}
\usepackage[brazil]{babel}
\usepackage{amsmath}
\usepackage{minted}

\title{Algoritmos e Programação II\\
\large Lista de Execícios: Funções}
\author{Prof. Evandro C. R. Rosa\\UNIVALI}
\date{}

\begin{document}
\maketitle

\noindent Nome Completo: \underline{\hspace{8.4cm}} Código de Pessoa: \underline{\hspace{2.4cm}}

\begin{enumerate}
    \item Responda sucintamente:
          \begin{enumerate}
              \item Por que variáveis locais perdem seus valores entre uma chamadas de função e outra?
              \item Onde você define variáveis de parâmetro?
              \item Se você estiver escrevendo uma função que aceita um argumento e deseja garantir que a função não possa alterar o valor do argumento, o que você deve fazer?
              \item Quando uma função aceita múltiplos argumentos, a ordem em que os argumentos são passados importa?
              \item Como você retorna um valor de uma função?
              \item Qual é a vantagem de dividir o código da sua aplicação em vários procedimentos (funções) pequenos?
              \item Dê um exemplo onde passar um argumento por referência seria útil.
              \item O que é necessário para que duas ou mais funções posam ter o mesmo nome?
          \end{enumerate}
    \item Os códigos abaixo possem um ou mais erros, indique quais:
          \begin{enumerate}
              \item \begin{minted}{cpp}
void total(int valor1, valor2, valor3) {
    return valor1 + valor2 + valor3;
}
        \end{minted}
              \item \begin{minted}{cpp}
double media(int valor1, int valor2, int valor3) {
    double media;
    media = valor1 + valor2 + valor3 / 3;
}
        \end{minted}
              \item \begin{minted}{cpp}
void area(int altura = 30, int largura) {
    return altura * largura;
}
        \end{minted}
              \item \begin{minted}{cpp}
void get_valor(int valor&) {
    std::cout << "Passe um valor: ";
    std::cin >> valor&;
}
        \end{minted}
          \end{enumerate}
    \item Escreva um programa que peça ao usuário para inserir o custo de atacado de um item e a porcentagem de margem de lucro. Em seguida, ele deve exibir o preço de varejo do item. Por exemplo:
          \begin{itemize}
              \item Se o custo de atacado de um item é 5,00 e a porcentagem de margem de lucro é 100\%, então o preço de varejo do item é 10,00.
              \item Se o custo de atacado de um item é 5,00 e a porcentagem de margem de lucro é 50\%, então o preço de varejo do item é 7,50.
          \end{itemize}

          O programa deve ter uma função chamada \texttt{calcularPrecoVarejo} que recebe o custo de atacado e a porcentagem de margem de lucro como argumentos e retorna o preço de varejo do item.

          \textbf{Validação de Entrada}: Não aceite valores negativos para o custo de atacado do item ou para a porcentagem de margem de lucro.


    \item Escreva um programa que determine qual das quatro filiais de uma empresa (Nordeste, Sudeste, Noroeste e Sudoeste) teve as maiores vendas em um trimestre. Ele deve incluir as seguintes duas funções, que são chamadas pela função \texttt{main}:
          \begin{itemize}
              \item  \texttt{double obterVendas(...)} recebe o nome de uma filial. Ele solicita ao usuário o valo total de vendas trimestrais da filial, valida a entrada e, em seguida, a retorna. Esta função deve ser chamada uma vez para cada filial.
              \item \texttt{void encontrarMaior(...)} recebe os quatro totais de vendas. Ela determina qual é o maior e imprime o nome da divisão com o maior faturamento, junto com seu valor de vendas.
          \end{itemize}

          \textbf{Validação de Entrada}: Não aceite valores em menores que 0,00.

    \item Um número primo é um número que só é divisível por ele mesmo e por 1. Por exemplo, o número 5 é primo porque só pode ser dividido por 1 e 5. No entanto, o número 6 não é primo porque pode ser dividido por 1, 2, 3 e 6. Escreva um programa que listar todos os números primos de 1 até um número \texttt{limite} fornecido pelo usuário. O programa deve incluir as seguintes funções:
          \begin{itemize}


              \item \texttt{bool ehPrimo(int numero)} recebe um número inteiro como argumento e retorna \texttt{true} se o número for primo, ou \texttt{false} caso contrário.

              \item  \texttt{void listarPrimos(int limite)} recebe um número inteiro \texttt{limite} e usa a função \texttt{ehPrimo} para listar todos os números primos de 1 até \texttt{limite}. A função deve imprimir os números primos em uma linha.
          \end{itemize}

          \textbf{Validação de Entrada}: Certifique-se de que o número fornecido pelo usuário é um inteiro positivo. Caso contrário, peça ao usuário para inserir um número válido.
    \item Uma empresa de pintura determinou que, para cada 10 metros quadrados de parede, é necessário um litro de tinta e oito horas de trabalho. A empresa cobra R\$25,00 por hora de trabalho. Escreva um programa modular que permita ao usuário inserir o número de cômodos a serem pintados e o preço da tinta por litro. O programa também deve solicitar a metragem quadrada das paredes de cada cômodo. Em seguida, deve exibir os seguintes dados:
          \begin{itemize}
              \item A quantidade de litros de tinta necessários
              \item As horas de trabalho necessárias
              \item O custo da tinta
              \item O custo do trabalho
              \item O custo total do serviço de pintura
          \end{itemize}

          \textbf{Validação de Entrada}: Não aceite um valor menor que 1 para o número de cômodos. Não aceite um valor menor que R\$10,00 para o preço da tinta. Não aceite um valor negativo para a metragem quadrada das paredes.

    \item Em uma competição de talentos, há cinco jurados, cada um atribuindo uma nota entre 0 e 10 para cada apresentador. Notas fracionárias, como 8,3, são permitidas. A nota final de um apresentador é determinada descartando a maior e a menor nota recebidas e, em seguida, calculando a média das três notas restantes. Escreva um programa que utilize esse método para calcular a nota de um competidor. O programa deve incluir as seguintes funções:

          \begin{itemize}
              \item  \texttt{void obterNotaJurado(...)} deve solicitar ao usuário a nota de um jurado, armazená-la em uma variável de parâmetro passada por referência e validar a entrada. A função deve ser chamada uma vez para cada um dos cinco jurados.
              \item  \texttt{void calcularNotaFinal(...)} deve calcular e exibir a média das três notas que permanecem após descartar as notas mais alta e mais baixa recebidas. Esta função deve ser chamada apenas uma vez pelo \texttt{main} e deve receber as cinco notas, além de usar as funções abaixo para o calculo.
                    \begin{itemize}
                        \item  \texttt{double encontrarMenor(...)} deve encontrar e retornar a menor das cinco notas passadas a ela.
                        \item  \texttt{double encontrarMaior(...)} deve encontrar e retornar a maior das cinco notas passadas a ela.
                    \end{itemize}
          \end{itemize}

          \textbf{Validação de Entrada}: Não aceite notas abaixo de 0 ou acima de 10.

    \item Suponha que você queira investir uma certa quantia de dinheiro no Tesouro Direto e deixá-la rendendo juros pelos próximos 10 anos. Ao final desses 10 anos, você gostaria de ter R\$10.000 investidos. Quanto você precisa investir hoje para que isso aconteça? Você pode usar a seguinte fórmula, conhecida como fórmula do valor presente, para descobrir:

          \begin{equation*}
              P = \frac{F}{(1 + r)^n}
          \end{equation*}

          Os termos da fórmula são os seguintes:
          \begin{itemize}
              \item $P$ é o valor presente, ou seja, a quantia que você precisa investir hoje.
              \item $F$ é o valor futuro que você deseja no investimento. (Neste caso, $F$ é R\$10.000.)
              \item $r$ é a taxa de juros anual.
              \item $n$ é o número de anos que você planeja deixar o dinheiro investido.
          \end{itemize}

          Escreva um programa que tenha uma função chamada \texttt{valorPresente}, que realiza esse cálculo. A função deve aceitar como argumentos o valor futuro, a taxa de juros anual e o número de anos. Ela deve retornar o valor presente, que é a quantia que você precisa investir hoje. Demonstre a função em um programa que permita ao usuário experimentar com diferentes valores para os termos da fórmula.

    \item Um estudante deseja calcular sua média final em uma disciplina, considerando três avaliações que possuem pesos diferentes. Escreva um programa que ajude o estudante a calcular sua média ponderada com base nas notas e pesos fornecidos. O programa deve incluir as seguintes funções:
          \begin{itemize}

              \item \texttt{void solicitarNotaEPeso(...)} solicita ao usuário a nota e o peso de uma avaliação e validando a entrada. Deve ser chamada para cada avaliação.

              \item \texttt{double calcularMediaPonderada(...)} recebe as três notas e seus respectivos pesos como argumentos e retorna a média ponderada calculada.
          \end{itemize}

          \textbf{Validação de Entrada}: Certifique-se de que as notas fornecidas estejam entre 0 e 10 e que os pesos sejam números positivos. Se o usuário fornecer valores inválidos, peça para inserir novamente até que os valores sejam corretos.

    \item Um sistema de gerenciamento de estoque precisa calcular o valor total de um pedido de produtos. Cada produto tem um preço unitário e uma quantidade solicitada. O programa deve calcular o valor total do pedido, aplicando um desconto se o valor total ultrapassar um determinado limite. Escreva um programa que realize esse cálculo. O programa deve incluir as seguintes funções:
          \begin{itemize}

              \item \texttt{void solicitarDadosProduto(...)} solicita ao usuário o preço unitário de um produto e a quantidade solicitada, validando a entrada para garantir que o preço seja positivo e a quantidade seja um número inteiro maior que zero.

              \item \texttt{double calcularValorTotal(..)} recebe o preço unitário e a quantidade solicitada e retorna o valor total do produto (preço $\times$ quantidade).

              \item \texttt{double aplicarDesconto(...)} aplica um desconto ao valor total se este ultrapassar o limite estabelecido (passado como argumento). A função retorna o valor total após o desconto (caso seja aplicável).

          \end{itemize}

          \textbf{Validação de Entrada}: Certifique-se de que o preço do produto seja positivo e que a quantidade seja um número inteiro maior que zero. Se o usuário fornecer valores inválidos, peça para inserir novamente até que os valores sejam corretos.
\end{enumerate}



\end{document}