\documentclass[12pt]{article}
\usepackage[a4paper, left=2cm, right=2cm, top=2cm, bottom=2cm]{geometry}
\usepackage[brazil]{babel}
\usepackage{amsmath}
\usepackage{minted}
\usepackage{caption}
\usepackage{subcaption}
\usepackage{graphicx}

\title{Algoritmos e Programação II\\
\large IV Lista de Exercícios: Ponteiros}
\author{Prof. Evandro C. R. Rosa\\UNIVALI}
\date{}

\begin{document}
\maketitle

\noindent Nome Completo: \underline{\hspace{8cm}} Código de Aluno: \underline{\hspace{2.4cm}}

\begin{enumerate}
    \item Responda sucintamente:
        \begin{enumerate}
            \item O que faz o operador de desreferência (\texttt{*})?
            \item Veja o código a seguir:
                \begin{minted}{cpp}
int x = 7;
int *iptr = &x; 
                \end{minted}
                \begin{enumerate}
                  \item O que será exibido se você enviar a expressão \texttt{*iptr} para \texttt{cout}?
                  \item E se você enviar a expressão \texttt{iptr} para \texttt{cout}?
                \end{enumerate}
            \item Quais operações matemáticas são permitidas com ponteiros?
            \item Qual é a finalidade do operador \texttt{new}?
            \item Qual é a finalidade do operador \texttt{delete}?
            \item Em quais circunstâncias é possível retornar um ponteiro de uma função?
        \end{enumerate}

    \item Os códigos abaixo contêm um ou mais erros. Indique quais são:
    \begin{enumerate}
      \item \begin{minted}{cpp}
  int ptr = nullptr;                  
      \end{minted}
      \item \begin{minted}{cpp}
  int x, *ptr = nullptr; 
  &x = ptr;
      \end{minted}
      \item \begin{minted}{cpp}
  int x, *ptr = nullptr;
  *ptr = &x;
      \end{minted}
      \item \begin{minted}{cpp}
  int x, *ptr = nullptr;
  ptr = &x;
  ptr = 100; // Armazena 100 em x  
  cout << x << endl;
      \end{minted}
      \item \begin{minted}{cpp}
  int numeros[] = {10, 20, 30, 40, 50};
  cout << "O terceiro elemento do array é ";
  cout << *numeros + 3 << endl;
      \end{minted}
      \item \begin{minted}{cpp}
  int valores[20], *ponteiro_int = nullptr;
  ponteiro_int = valores;
  ponteiro_int *= 2;
      \end{minted}
      \item \begin{minted}{cpp}
  float nivel;
  int ponteiro_float = &nivel;
      \end{minted}
      \item \begin{minted}{cpp}
  int *ponteiro_int = &numero;
  int numero;
      \end{minted}
      \item \begin{minted}{cpp}
  void dobrar_valor(int valor)  {
      *valor *= 2;
  }
      \end{minted}
      \item \begin{minted}{cpp}
  int *ponteiro_int = nullptr; 
  ponteiro_int = new int[100]; // Aloca memória
  delete ponteiro_int; // Libera memória
      \end{minted}
      \item \begin{minted}{cpp}
  int *obter_numero()  {
      int numero;  
      cout << "Digite um número: "; 
      cin >> numero; 
      return &numero;  
  }
      \end{minted}
      \item \begin{minted}{cpp}
  struct TresValores {  
      int a, b, c; 
  };
  int main () {  
      TresValores s, *s_ponteiro = nullptr; 
      s_ponteiro = &s; 
      *s_ponteiro.a = 1; 
      return 0;
  }
      \end{minted}
  \end{enumerate}
  

    \item Em estatística, a moda de um conjunto de valores é o valor que ocorre com maior frequência. Escreva uma função que aceite os seguintes argumentos: 
    \begin{itemize}
      \item Um array de inteiros
      \item Um inteiro que indica o número de elementos no array
    \end{itemize}
    A função deve determinar a moda do array, ou seja, qual valor no array ocorre com mais frequência. A moda é o valor que a função deve retornar. Se o array não tiver moda (nenhum valor ocorre mais de uma vez), a função deve retornar -1 -- Assuma que o array sempre conterá valores positivos. Demonstre a função em um programa completo.

    \item Escreva uma função que aceite um array de inteiros e o tamanho do array como argumentos. A função deve criar uma cópia do array, exceto que os valores dos elementos devem ser invertidos na cópia. A função deve retornar um ponteiro para o novo array. Demonstre a função em um programa completo.
    
    \item Escreva uma função que aceite um array de inteiros e o tamanho do array como argumentos. A função deve criar um novo array que seja o dobro do tamanho do array argumento. A função deve copiar o conteúdo do array argumento para o novo array e inicializar os elementos não utilizados do segundo array com 0. A função deve retornar um ponteiro para o novo array. Demonstre a função em um programa completo.

    \item Escreva um programa que aloca dinamicamente um array grande o suficiente para armazenar uma quantidade de notas fornecida pelo usuário. Após a inserção de todas as notas (de 0 a 10), o array deve ser passado para uma função que as ordena em ordem crescente. Outra função deve ser chamada para calcular a média das notas. O programa deve exibir a lista ordenada de notas e a média com títulos apropriados. 

    \item Modifique o problema acima para que a menor nota seja descartada. Essa nota não deve ser incluída no cálculo da média.

    \item Modifique o programa do problema acima para permitir que o usuário insira pares nome-nota. Para cada aluno que fizer uma prova, o usuário deve digitar o nome do aluno seguido pela nota da prova do aluno. Modifique a função de ordenação para que ela aceite um array contendo os nomes dos alunos e um array contendo as notas dos testes dos alunos. Quando a lista ordenada de notas for exibida, o nome de cada aluno deve ser exibido junto com sua nota.


    \item Escreva um programa que use uma estrutura para armazenar os seguintes dados:
        \begin{itemize}
            \item nome: Nome do aluno
            \item notas: Ponteiro para um array de notas
            \item media: Média das notas
        \end{itemize}
        O programa deve manter uma lista de notas para um grupo de alunos. Deve perguntar ao usuário quantas notas há e quantos alunos existem. Em seguida, deve alocar dinamicamente um array de estruturas. O membro \texttt{notas} de cada estrutura deve apontar para um array alocado dinamicamente que conterá as notas. Após a alocação dinâmica dos arrays, o programa deve solicitar o nome e todas as notas para cada aluno. A média das notas deve ser calculada e armazenada no membro \texttt{media} de cada estrutura. Depois que todos esses dados forem calculados, uma tabela deve ser exibida na tela listando o nome de cada aluno e a média das notas. 

        \textbf{Validação de Entrada}: Não aceite números negativos para nenhuma nota.

\end{enumerate}

\end{document}
