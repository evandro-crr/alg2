\documentclass[12pt]{article}
\usepackage[a4paper, left=2cm, right=2cm, top=2cm, bottom=2cm]{geometry}
\usepackage[brazil]{babel}
\usepackage{amsmath}
\usepackage{minted}
\usepackage{caption}
\usepackage{subcaption}
\usepackage{graphicx}

\title{Algoritmos e Programação II\\
\large VI Lista de Exercícios: Introdução a Classes}
\author{Prof. Evandro C. R. Rosa\\UNIVALI}
\date{}

\begin{document}
\maketitle

\noindent Nome Completo: \underline{\hspace{8cm}} Código de Aluno: \underline{\hspace{2.4cm}}

\begin{enumerate}
  \item Responda sucintamente:
    \begin{enumerate}
      \item Qual é a diferença entre uma classe e uma instância da classe?
      \item Qual é a diferença entre a estrutura \texttt{Pessoa} e a classe \texttt{Pessoa}, como mostrado abaixo?
      
        \begin{minipage}{.3\linewidth}
          \begin{minted}{cpp}
struct Pessoa {
    string nome;
    int idade;
};
          \end{minted}
        \end{minipage}
        \begin{minipage}{.3\linewidth}
          \begin{minted}{cpp}
class Pessoa {
    string nome;
    int idade;
};
          \end{minted}
        \end{minipage}

      \item Qual é a especificação de acesso padrão dos membros de uma classe?
      \item Veja a definição da seguinte função membro:\\ \mintinline{cpp}|void Circulo::obterRaio()|
        \begin{enumerate}
          \item Qual é o nome da função?
          \item De qual classe a função é membro?
        \end{enumerate}
      \item É uma boa ideia tornar as variáveis membros privadas?
      \item Em quais circunstâncias uma função membro deve ser privada?
      \item O que é um construtor? O que é um destrutor?
      \item O que é um construtor padrão? É possível ter mais de um construtor padrão?
      \item É possível ter mais de um construtor? E mais de um destrutor?
    \end{enumerate}

  \item Os códigos abaixo contêm um ou mais erros. Indique quais são:
    \begin{enumerate}
      \item
        \begin{minted}{cpp}
class Circulo: {
    private
        double centroX;
        double centroY;
        double raio;
    public
        definirCentro(double, double);
        definirRaio(double);
}
        \end{minted}

        \newpage
      \item
        \begin{minted}{cpp}
#include <iostream>
using namespace std;

class Haltere;
{
    int peso;

  public:
    void definirPeso(int);
};

void definirPeso(int p) { peso = p; }

int main() {
    Haltere barra;
    Haltere(200);
    cout << "O peso é " << barra.peso << endl;
    return 0;
}
        \end{minted}
      \item
        \begin{minted}{cpp}
class Troco {
  public:
    int centavos;
    int cinco_centavos;
    int dez_centavos;
    int vinte_cinco_centavos;
    Troco() {
        centavos = cincoCentavos = dezCentavos = vinteCincoCentavos = 0;
    }
    Troco(int c = 100, int cc = 50, d = 50, vcc = 25);
};

void Troco::Troco(int c, int cc, d, vcc) {
    centavos = c;
    cincoCentavos = cc;
    dezCentavos = d;
    vinteCincoCentavos = vcc;
}
        \end{minted}
    \end{enumerate}

  \item Crie uma classe chamada \texttt{Data}. A classe deve armazenar uma data em três inteiros: mês, dia e ano. Devem haver funções membros para imprimir a data nos seguintes formatos:
    \begin{itemize}
      \item 25/12/2024
      \item 25 de dezembro de 2024
    \end{itemize}
    Demonstre a classe escrevendo um programa completo que a implemente. Validação de entrada: Não aceite valores para o dia maiores que 31 ou menores que 1. Não aceite valores para o mês maiores que 12 ou menores que 1.

  \item Crie uma classe chamada \texttt{Estoque} que armazene informações e calcule dados sobre itens no inventário de uma loja. A classe deve ter as seguintes variáveis privadas:

    \begin{tabular}{|l|p{.7\linewidth}|}
      \hline
      Nome da Variável & Descrição \\
      \hline
      \texttt{codigo\_item} & Um int que armazena o código do item. \\
      \texttt{quantidade} & Um int que armazena a quantidade de itens em estoque. \\
      \texttt{custo} & Um double que armazena o custo por unidade do item. \\
      \texttt{custo\_total} & Um double que armazena o custo total do item (calculado como quantidade vezes custo). \\
      \hline
    \end{tabular}

    A classe deve ter as seguintes funções públicas:
    \begin{itemize}
      \item Construtor Padrão: Define todas as variáveis membro como 0.
      \item Construtor: Aceita como argumentos o código, o custo e a quantidade de um item. A função deve copiar esses valores para as variáveis membro apropriadas e então chamar a função \texttt{definir\_custo\_total}.
      \item \texttt{definir\_codigo\_item}: Aceita um argumento do tipo inteiro que é copiado para a variável membro \texttt{codigo\_item}.
      \item \texttt{definir\_quantidade}: Aceita um argumento do tipo inteiro que é copiado para a variável membro quantidade.
      \item \texttt{definir\_custo}: Aceita um argumento do tipo double que é copiado para a variável membro custo.
      \item \texttt{definir\_custo\_total}: Calcula o custo total do item no inventário (quantidade vezes custo) e armazena o resultado em \texttt{custo\_total}.
      \item \texttt{obter\_codigo\_item}: Retorna o valor de \texttt{codigo\_item}.
      \item \texttt{obter\_quantidade}: Retorna o valor de quantidade.
      \item \texttt{obter\_custo}: Retorna o valor de custo.
      \item \texttt{obter\_custo\_total}: Retorna o valor de \texttt{custo\_total}.
    \end{itemize}
    Demonstre a classe em um programa.

    \textbf{Validação de Entrada}: Não aceite valores negativos para código, quantidade ou custo.

  \item Crie uma classe chamada \texttt{NotasProva} que tenha variáveis membro para armazenar três notas de prova. A classe deve ter um construtor, funções de acesso e de modificação para as notas, e uma função membro que retorne a média das notas. Demonstre a classe escrevendo um programa separado que crie uma instância da classe e peça ao usuário para inserir três notas de prova, que devem ser armazenadas no objeto \texttt{NotasProva}. O programa deve exibir a média das notas, conforme relatado pelo objeto \texttt{NotasProva}.

  \item Escreva uma classe chamada \texttt{Circulo} que tenha as seguintes variáveis de membro:
    \begin{itemize}
      \item \texttt{raio}: um \texttt{double}
      \item \texttt{pi}: um \texttt{double} inicializado com o valor 3.14159
    \end{itemize}
    A classe deve ter as seguintes funções membro:
    \begin{itemize}
      \item Construtor Padrão: um construtor padrão que define o raio como 0.0.
      \item Construtor: aceita o raio do círculo como argumento.
      \item \texttt{set\_raio}: uma função mutadora para a variável raio.
      \item \texttt{get\_raio}: uma função acessora para a variável raio.
      \item \texttt{get\_area}: retorna a área do círculo, calculada como área $= \text{pi} \times \text{raio}^2$.
      \item \texttt{get\_diametro}: retorna o diâmetro do círculo, calculado como diâmetro $= 2\text{raio}$.
      \item \texttt{get\_circunferencia}: retorna a circunferência do círculo, calculada como circunferência $= 2\text{pi} \times \text{raio}$.
    \end{itemize}
    Escreva um programa que demonstre a classe Círculo, solicitando ao usuário o raio do círculo, criando um objeto Círculo e, em seguida, exibindo a área, o diâmetro e a circunferência do círculo.

  \item Projete uma classe que tenha um array de números de ponto flutuante. O construtor deve aceitar um argumento inteiro e alocar dinamicamente o array para armazenar essa quantidade de números. O destrutor deve liberar a memória ocupada pelo array. Além disso, deve haver funções membro para realizar as seguintes operações:
    \begin{itemize}
      \item Armazenar um número em qualquer elemento do array
      \item Recuperar um número de qualquer elemento do array
      \item Retornar o maior valor armazenado no array
      \item Retornar o menor valor armazenado no array
      \item Retornar a média de todos os números armazenados no array
    \end{itemize}
    Demonstre a classe em um programa.

  \item Escreva uma classe chamada \texttt{Moeda} que deve ter a seguinte variável membro:
    \begin{itemize}
      \item Uma string chamada \texttt{lado\_para\_cima} que deve armazenar ``cara'' ou ``coroa'', indicando qual lado da moeda está voltado para cima.
    \end{itemize}
    A classe Moeda deve ter as seguintes funções membro:
    \begin{itemize}
      \item Um construtor padrão que determina aleatoriamente qual lado da moeda está voltado para cima (``cara'' ou ``coroa'') e inicializa a variável \texttt{lado\_para\_cima} de acordo.
      \item Uma função membro \texttt{jogar} que simula o lançamento da moeda. Quando a função \texttt{jogar} é chamada, ela determina aleatoriamente qual lado da moeda está voltado para cima e define a variável \texttt{lado\_para\_cima} de acordo.
      \item Uma função membro chamada \texttt{get\_lado\_para\_cima} que retorna o valor da variável \texttt{lado\_para\_cima}.
    \end{itemize}
    Escreva um programa que demonstre a classe Moeda. O programa deve criar uma instância da classe e exibir o lado que está inicialmente voltado para cima. Em seguida, use um loop para lançar a moeda 20 vezes. Cada vez que a moeda for lançada, exiba o lado que está voltado para cima. O programa deve contar o número de vezes que ``cara'' está voltado para cima e o número de vezes que ``coroa'' está voltado para cima, e exibir esses valores após o loop.

\end{enumerate}

\end{document}
