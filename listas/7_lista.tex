\documentclass[12pt]{article}
\usepackage[a4paper, left=2cm, right=2cm, top=2cm, bottom=2cm]{geometry}
\usepackage[brazil]{babel}
\usepackage{amsmath}
\usepackage{minted}
\usepackage{caption}
\usepackage{subcaption}
\usepackage{graphicx}

\title{Algoritmos e Programação II\\
\large VI Lista de Exercícios: Manipulação de Arquivos}
\author{Prof. Evandro C. R. Rosa\\UNIVALI}
\date{}

\begin{document}
\maketitle

\noindent Nome Completo: \underline{\hspace{8cm}} Matrícula: \underline{\hspace{2.4cm}}

\begin{enumerate}
  \item Os códigos abaixo contêm um ou mais erros. Identifique-os e explique como corrigi-los:
    \begin{enumerate}
      \item
        \begin{minted}{cpp}
fstream file(ios::in | ios::out);
file.open("info.dat");
if (!file) {
    cout << "Não foi possível abrir o arquivo.\n";
}
        \end{minted}
      \item
        \begin{minted}{cpp}
ofstream file;
file.open("info.dat", ios::in);
if (file) {
    cout << "Não foi possível abrir o arquivo.\n";
}
        \end{minted}
      \item
        \begin{minted}{cpp}
fstream data_file("info.dat", ios::in | ios::binary);
int x = 5;
data_file << x;
        \end{minted}
      \item
        \begin{minted}{cpp}
fstream data_file("info.dat", ios::in);
char stuff[81];
data_file.get(stuff);
        \end{minted}
      \item
        \begin{minted}{cpp}
fstream data_file("info.dat", ios::in);
char stuff[81] = "abcdefghijklmnopqrstuvwxyz";
data_file.put(stuff);
        \end{minted}
      \item
        \begin{minted}{cpp}
fstream data_file("info.dat", ios::out);
struct Date {  int month; int day; int year; };
Date dt = { 4, 2, 98 };
data_file.write(&dt, sizeof(int));
        \end{minted}
    \end{enumerate}

  \item Escreva um programa que solicite ao usuário o nome de um arquivo. O programa deve exibir as 10 primeiras linhas do arquivo na tela. Se o arquivo tiver menos de 10 linhas, o conteúdo inteiro deve ser exibido, com uma mensagem indicando que todo o arquivo foi mostrado.

  \item Escreva um programa que solicite ao usuário o nome de um arquivo. O programa deve exibir o conteúdo do arquivo na tela. Se o conteúdo do arquivo não couber em 24 linhas, o programa deve exibir 24 linhas por vez e depois pausar. A cada pausa, o programa deve esperar o usuário pressionar uma tecla para exibir as próximas 24 linhas.

  \item Escreva um programa que solicite ao usuário o nome de um arquivo. O programa deve exibir as últimas 10 linhas do arquivo na tela. Se o arquivo tiver menos de 10 linhas, o conteúdo inteiro deve ser exibido, com uma mensagem indicando que todo o arquivo foi mostrado.

  \item Escreva um programa que solicite ao usuário o nome de um arquivo. O programa deve exibir o conteúdo do arquivo na tela. Cada linha exibida deve ser precedida de um número de linha, seguido por dois pontos. A numeração das linhas deve começar em 1. Veja o exemplo:

  \texttt{1: Olá\\
  2: Esse é um exemplo de arquivo\\
  3: com apenas 3 linhas.}

  Se o conteúdo do arquivo não couber em uma única tela, o programa deve exibir 24 linhas por vez e depois pausar. A cada pausa, o programa deve esperar o usuário pressionar uma tecla para exibir as próximas 24 linhas.

  \item Escreva um programa que solicite ao usuário o nome de um arquivo e uma string para busca. O programa deve procurar no arquivo todas as ocorrências da string especificada. Quando a string for encontrada, a linha que a contém deve ser exibida. Após todas as ocorrências serem localizadas, o programa deve informar quantas vezes a string apareceu no arquivo.

  \item Escreva uma função chamada \texttt{array\_para\_arquivo}. A função deve aceitar três argumentos: o nome de um arquivo, um ponteiro para um array de inteiros e o tamanho do array. A função deve abrir o arquivo especificado no modo binário, escrever o conteúdo do array no arquivo e depois fechar o arquivo. Escreva outra função chamada \texttt{arquivo\_para\_array}. Esta função deve aceitar três argumentos: o nome de um arquivo, um ponteiro para um array de inteiros e o tamanho do array. A função deve abrir o arquivo no modo binário, ler seu conteúdo para o array e depois fechar o arquivo. Escreva um programa completo que demonstre essas funções, utilizando \texttt{array\_para\_arquivo} para gravar um array em um arquivo, e em seguida, usando \texttt{arquivo\_para\_array} para ler os dados do mesmo arquivo. Após ler os dados, exiba o conteúdo do array na tela.

  \item Criptografia de arquivos é a ciência de escrever o conteúdo de um arquivo em código secreto. Seu programa de criptografia deve funcionar como um filtro, lendo o conteúdo de um arquivo, modificando os dados para um código e escrevendo o conteúdo codificado em um segundo arquivo. O segundo arquivo será uma versão do primeiro, mas escrito em código secreto. Embora existam técnicas de criptografia complexas, você deve criar uma simples. Por exemplo, você pode ler o primeiro arquivo um caractere por vez e adicionar 10 ao código ASCII de cada caractere antes de escrevê-lo no segundo arquivo.

  \item Escreva um programa que descriptografe o arquivo produzido pelo programa do exercício anterior. O programa de descriptografia deve ler o conteúdo do arquivo codificado, restaurar os dados ao seu estado original e escrevê-los em outro arquivo.
  
\end{enumerate}

\end{document}
