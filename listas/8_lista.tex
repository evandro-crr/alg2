\documentclass[12pt]{article}
\usepackage[a4paper, left=2cm, right=2cm, top=2cm, bottom=2cm]{geometry}
\usepackage[brazil]{babel}
\usepackage{amsmath}
\usepackage{minted}
\usepackage{caption}
\usepackage{subcaption}
\usepackage{graphicx}

\title{Algoritmos e Programação II\\
\large VIII Lista de Exercícios: Manipulação de Arquivos}
\author{Prof. Evandro C. R. Rosa\\UNIVALI}
\date{}

\begin{document}
\maketitle

\noindent Nome Completo: \underline{\hspace{8cm}} Matrícula: \underline{\hspace{2.4cm}}

\begin{enumerate}
  \item Desenvolva um programa de gerenciamento de estoque, onde cada item possui as seguintes informações:
    \begin{minted}{cpp}
struct Produto {
  char codigo[8];
  char nome[50];
  char descricao[250];
  double valor_unitario;
  int quantidade;
};
    \end{minted}

    O programa deve oferecer as seguintes funcionalidades:
    \begin{itemize}
      \item Cadastro de novos produtos;
      \item Atualização dos dados de um produto existente;
      \item Exclusão de produtos do estoque;
      \item Geração de relatórios detalhados dos produtos cadastrados;
    \end{itemize}

    Além disso, todos os dados devem ser armazenados em um arquivo, permitindo que o usuário carregue as informações do estoque ao iniciar o programa.

  \item No exercício anterior, o nome e a descrição do produto possuem um tamanho fixo, definido em tempo de compilação. Isso simplifica a manipulação dos dados no arquivo, porém limita a quantidade de informações que podem ser armazenadas por produto, além de potencialmente aumentar o tamanho do arquivo.

    Adapte o programa anterior para utilizar a seguinte estrutura, onde o nome e a descrição do produto são alocados dinamicamente:
    \begin{minted}{cpp}
struct Produto {
  char codigo[8];
  char *nome;
  char *descricao;
  double valor_unitario;
  int quantidade;
};
    \end{minted}

    Essa modificação deve permitir o uso eficiente da memória, ajustando o espaço alocado conforme a quantidade de dados inseridos pelo usuário.
\end{enumerate}

\end{document}
