\documentclass[12pt]{article}
\usepackage[a4paper, left=2cm, right=2cm, top=2cm, bottom=2cm]{geometry}
\usepackage[brazil]{babel}
\usepackage{amsmath}
\usepackage{minted}
\usepackage{caption}
\usepackage{subcaption}
\usepackage{graphicx}

\title{Algoritmos e Programação II\\
\large IX Lista de Exercícios: Recursão}
\author{Prof. Evandro C. R. Rosa\\UNIVALI}
\date{}

\begin{document}
\maketitle

\noindent Nome Completo: \underline{\hspace{8cm}} Matrícula: \underline{\hspace{2.4cm}}
\begin{enumerate}
  \item Escreva uma função recursiva que retorne o número de vezes que um determinado número ocorre em um array.
  \item Escreva uma função recursiva que retorne o maior valor de um array.
  \item Escreva uma função que aceite um array de inteiros e um número indicando o número de elementos como argumentos. A função deve calcular recursivamente a soma de todos os números no array. Demonstre a função em um programa.
  \item Escreva uma função recursiva que aceite dois argumentos nos parâmetros \( x \) e \( y \). A função deve retornar o valor de \( x \) multiplicado por \( y \). Lembre-se de que a multiplicação pode ser realizada como adição repetida: \( 7 \times 4 = 4 + 4 + 4 + 4 + 4 + 4 + 4 \).
  \item Escreva uma função que use recursão para elevar um número a uma potência. A função deve aceitar dois argumentos: o número a ser elevado e o expoente. Assuma que o expoente é um inteiro não negativo. Demonstre a função em um programa.
  \item Escreva uma função que aceite um número inteiro como argumento e retorne a soma de todos os inteiros de 1 até o número passado como argumento. Por exemplo, se 50 for passado como argumento, a função retornará a soma de 1, 2, 3, 4, ..., 50. Use recursão para calcular a soma. Demonstre a função em um programa.
  \item Escreva uma função recursiva booleana que aceite dois argumentos: um array e um valor. A função deve retornar verdadeiro se o valor for encontrado no array, ou falso se o valor não for encontrado no array. Demonstre a função em um programa.
  \item Escreva uma função recursiva que aceite uma string como argumento e imprima a string em ordem inversa. Demonstre a função em um programa.
\end{enumerate}


\end{document}
