\documentclass[12pt]{article}
\usepackage[a4paper, left=2cm, right=2cm, top=2cm, bottom=2cm]{geometry}
\usepackage[brazil]{babel}
\usepackage{amsmath}
\usepackage{minted}
\usepackage{caption}
\usepackage{subcaption}
\usepackage{graphicx}
\usepackage{enumitem}

\title{Algoritmos e Programação II\\
\large Trabalho 3: Leitura de Metadados de Arquivos Binários}
\author{Prof. Evandro C. R. Rosa\\UNIVALI}
\date{}

\begin{document}
\maketitle

\section*{Objetivo}
Em equipes de até 3 alunos, desenvolvam um programa em C++ que seja capaz de ler e extrair os metadados de um arquivo binário. Exemplos de arquivos binários incluem, mas não estão limitados a, arquivos de áudio, vídeo, imagem, entre outros.

\section*{Descrição da Atividade}
Cada equipe deverá escolher um formato de arquivo binário (por exemplo, MP3, WAV, JPEG, PNG, MP4, AVI, etc.) e estudar como esse formato organiza e armazena seus metadados. Com base nesse estudo, o grupo deverá desenvolver um programa que:

\begin{itemize}
  \item Carregue o arquivo binário usando \texttt{fstream} (sem utilizar bibliotecas externas, apenas a STL do C++).
  \item Extraia e exiba na tela os metadados presentes no arquivo, como título, autor, resolução, duração, bitrate, entre outros que forem aplicáveis ao formato escolhido.
\end{itemize}

\section*{Requisitos do Programa}
\begin{itemize}
  \item O programa deve ser implementado em \textbf{C++}, utilizando a biblioteca padrão (\texttt{fstream} para manipulação de arquivos e outras ferramentas da STL quando necessário).
  \item O uso de bibliotecas externas está \textbf{proibido}. Todo o processamento e extração de metadados deve ser feito manualmente através da leitura direta dos dados binários.
  \item A equipe deverá garantir que o programa seja capaz de lidar com diferentes arquivos do formato escolhido, validando adequadamente a entrada.
\end{itemize}

\section*{Entregáveis}
\begin{enumerate}
  \item \textbf{Código-fonte do Programa}:
    \begin{itemize}
      \item O código deve ser entregue até a data das apresentações.
      \item O código deve estar bem estruturado e comentado, explicando as estratégias utilizadas para a leitura e extração dos metadados.
    \end{itemize}
    \newpage

  \item \textbf{Apresentação Oral}:
    \begin{itemize}
      \item Cada grupo deverá realizar uma apresentação de \textbf{10 minutos}. A apresentação deve conter:
        \begin{itemize}
          \item Uma explicação sobre como os metadados são estruturados no formato específico de arquivo escolhido pelo grupo.
          \item As estratégias adotadas para extrair esses metadados utilizando a manipulação de arquivos binários em C++.
          \item Uma demonstração prática do programa, mostrando como ele é capaz de ler e exibir os metadados de um exemplo de arquivo.
        \end{itemize}
      \item \textbf{Obs}: Não é necessário entregar o material da apresentação, apenas realizá-la no dia definido.
    \end{itemize}
\end{enumerate}


\section*{Critérios de Avaliação}
\begin{itemize}
  \item \textbf{Funcionalidade}: O programa deve ser capaz de extrair corretamente os metadados do formato de arquivo escolhido.
  \item \textbf{Organização do código}: O código deve ser bem estruturado, modularizado e comentado.
  \item \textbf{Criatividade e Complexidade}: Serão levados em consideração o formato escolhido e a complexidade dos metadados extraídos.
  \item \textbf{Qualidade da apresentação}: Clareza na explicação da estrutura dos metadados e do código, além da qualidade da demonstração prática.
\end{itemize}

\end{document}
